\documentclass[a4paper]{article}
\usepackage{lingmacros}
\usepackage{tree-dvips}
\usepackage[T1]{fontenc}
\usepackage[utf8]{inputenc}

\usepackage[romanian]{babel}
\usepackage{combelow}
\usepackage{newunicodechar}


\begin{document}
\title{\textbf{{Descriere licență}{\Large }}}
\title{Descriere licență}
\author{Ciucanu Niculiță-Laurențiu}
\date{}
\maketitle

\textbf{Context}\\

În ziua de astăzi timpul reprezintă atât un prieten cât și un dușman al omului depinzând de modul de organizare al fiecăruia. Ai observat că în zilele noastre timpul nu-ți este prieten mai deloc? De multe ori persoanele așteaptă prea mult timp tramvaiele sau autobuzele fără să știe exact când acestea vor veni. Uneori acest timp poate ajunge la zeci de minute. Utilitatea acestei aplicații este de a diminua timpul de așteptare al călătorilor și de a fi informați în timp real despre locul unde se află mijlocul de transport în comun așteptat și timpul aproximativ în care va ajunge în stație. Timpul fiind foarte important, acum nu va mai fi irosit așteptând în stație. Orice persoană s-a confrundat cu acestă situație cel puțin odată, deci crearea unei aplicații Android ar rezolva această problemă. 
\newline
\newline

\textbf{Motivație}\\

Această lucrare are scopul de a realiza o aplicație care ajută persoanele din orașul Iași să fie informate în timp real unde se află mijlocul de transport(tramvai sau autobuz) pe care îl caută.
Cu această aplicație poți ști exact unde se află mijlocul de transport în comun pe care îl aștepți. De asemenea, o altă utilitate a acestei aplicații este găsirea celui mai scurt traseu până la destinație. Utilizatorul poate caută loc unde vrea să ajungă și aplicație va găsi traseul cel mai optim. În afară de folosirea mijloacelor de transport în comun se va afișa și un traseu adițional de mers pe jos de la locul unde se află utilizatorul  până la cea mai apropiată stație. O altă caracteristică a acestei aplicații este că utilizatorii pot adăuga întârzieri și motivul acestora. Toți ceilalți utilizatori vor vedea apariția întârzierii printr-o notificare. Astfel toții utilizatorii care urmăresc traficul de pe un anumit traseu vor ști din timp că au apărut întârzieri. Traseele care sunt afectate de această problemă vor fi marcate cu culoarea roșie și un timp aproximativ de așteptare în plus.


\end{document}